% This file belongs to the CWEB-NLS package.


\def\getCVSrevision$#1: #2 ${\def\RCSrevision{#2}}
\getCVSrevision$Revision: 11 $
\message{Loading CWEB-NLS utility (revision: \RCSrevision)}

\catcode`\@=\catcode`\A

% Define the basic error-reporting and abort mechanisms 
% (borrowed from etex.src)

\def\n{^^J}
\newlinechar=\expandafter`\n 

\def\et@xmsg #1#2%
   {\begingroup
    \def\ { }
    \message{!\if F#1%
              \else
                unknown (#1)%
              \fi\ #2}
    \endgroup}

\def\et@xabort #1%
   {\et@xmsg{F}{#1}
    \batchmode
    \end
   }

% modifed loop: \LOOP .. \if.. .. \else .. \REPEAT
\long
\def\LOOP #1\REPEAT{\def\b@dy{#1}\it@rat@}
\def\it@rat@ {\b@dy \expandafter \it@rat@ \fi}
\let\REPEAT=\fi

% the macro `\lang' should be defined on the command line
\ifx\undefined\lang
   \et@xabort
     {Usage: tex '\string\def\string\lang{<country code (eg. pl_PL)>}
                  \string\input\ cweb-nls'\n}
\fi

% read line: `<number> <translation>' and define control sequence
% \MESSAGE:<number> expanding to <translation>

\def\TEMPLATEfilename{template.tex}

\newread\msg 
\openin\msg=\lang.msg
\newwrite\translat@d
\immediate\openout\translat@d=\lang-cweb.tex
\newwrite\driverfil@
\immediate\openout\driverfil@=cweb-\lang.tex
\newread\templat@
\openin\templat@=\TEMPLATEfilename
\newread\xd@fs
\openin\xd@fs=\lang.def

\let\msgch@r=: % each line with translated messages should begin with `:'

\def\ch@ck{\futurelet\ch@r\pars@}
\def\pars@
   {\ifx\msgch@r\ch@r
      \let\next=\parselin@
    \else
      \let\next=\skiplin@
    \fi
    \next}
\def\parselin@:#1:#2\@nd
   {\expandafter\gdef\csname MESSAGE:#1\endcsname{#2}}
\def\skiplin@#1\@nd{}

% Read files :
%    <country code>.tex, template.tex and (possibly) <country code>.def
% and write file
%    cweb-<country code>.tex
% with macros for CWEB-NLS

% Read <country code>.tex file and define `\MESSAGE:<number>' macros
% expanding to translated messages.
\ifeof\msg 
   \et@xabort
      {File with translated messages `\lang.tex' does not exist.}
\else
   \begingroup \endlinechar=-1
     \LOOP 
     \ifeof\msg
     \else
        \read\msg to \lin@
        \expandafter\ch@ck\lin@\@nd
     \REPEAT
   \endgroup
\fi
\closein\msg

% Write cweb-<country code>.tex file
\begingroup 
  \def\MESSAGE#1{\csname MESSAGE:#1\endcsname}
  \begingroup% `^' should be introduced via `^CIRCUMFLEX'
    \catcode`\^=12 
    \gdef\CIRCUMFLEX{^}
  \endgroup
  \catcode`\^=0% make `^' an escape character
  \catcode`\#=12% make `#' category other
  \catcode`\~=12% make `~' category other
  \endlinechar=-1% do not put ` ' at the end of read line
  \catcode`\%=12% write comment lines and make `\' category other
  \catcode`\\=12
  ^immediate^write^driverfil@{% This file was automatically generated by TeX.}
  ^immediate^write^driverfil@{% This file belongs to the CWEB-NLS package.}
  ^immediate^write^driverfil@{% Copyright (C) 2001 Wlodek Bzyl}
  ^immediate^write^driverfil@{\input C/cweb-mac.tex}
  ^immediate^write^driverfil@{\input ^lang/messages.tex}
  ^ifeof^templat@
     ^et@xabort{Can not find file: `^TEMPLATEfilename'.}
  ^else
    ^LOOP
    ^ifeof^templat@
    ^else
      ^read^templat@ to ^lin@
      ^immediate^write^translat@d{^lin@}
    ^REPEAT
  ^fi
  ^ifeof^xd@fs
  ^else
    ^LOOP
    ^ifeof^xd@fs
    ^else
      ^read^xd@fs to ^lin@
      ^immediate^write^translat@d{^lin@}
    ^REPEAT
  ^fi
^endgroup

\closein\templat@
\closein\xd@fs

\end
